\documentclass[10pt,a4paper]{article}
\usepackage[utf8]{inputenc}
\usepackage[spanish]{babel}
\usepackage{amsmath}
\usepackage{amsfonts}
\usepackage{amssymb}
\usepackage{graphicx}
\usepackage{listings}

\begin{document}
\begin{large}
\textbf{Tarea 4}
\end{large}

\section{Pregunta 01}

\begin{itemize}
\item[A] Factoriza 35 tratando de calcular 3Q.\vspace{.3cm}

Primero para poder realizar esta operación tenemos que enontrar 2Q, para ello tenemos que sumar los puntos (15,4) + (15,4) Recordando que la suma se define de la siguiente forma\vspace{.3cm}

P+Q=\[
\begin{cases}
Infinito  & \text{SI }  x_1=x_2 \ \&  -y_1= y_2 ;\\
(x_3,y_3) &  \text{} x_3= (\lambda -x_1 - x_2)\  mod \ p \  y\  y_3 = (\lambda (x_1-x_3)-x_1)\ mod \ p\\

\end{cases}
\]  
Pero para esto necesitamos sacar primero a $\lambda$ que recordemos que se difine de la siguiente manera\vspace{.3cm}

 $\lambda$=\[
\begin{cases}
((3x_1^2+A) *  2(y_1)^{-1})) \ mod \ p & \text{si } P=Q ;\\
((y1-y2)*(x_1 - x_2))\ mod \ p & \text{si } P!=Q;\\

\end{cases}
\]

Para esto entonces simplemente sustituimos los valores, en lambda debido a que P=Q entonces usamos el primer caso de la lamda lo cual nos dice que $\lambda$ =
$(((3(15)^2)+-20) *  2(-4)^{-1}))mod 35$\\
$\rightarrow$ (((3*225)-20* (-8)$^{-1}$)mod 35\\
$\rightarrow$  ((675-20)*13)mod 35\\
 $\rightarrow$ (655 * 13) mod 35\\
  $\rightarrow$ 8515 mod 35\\
   $\therefore$ $\lambda = $  10\vspace{.3cm} 
   
 Ahora ya podemos sumar, primero sacaremos $x_3$ = $10^2$-15-15 mod 35 \\
 $\rightarrow$ 100-30 mod 35\\
 $\rightarrow$70 mod 35\\
 $\therefore$ x3=0\vspace{.3cm}
 
 Ahora debemos sacar a $y_3$=(10(15-0)- -4) mod 35 \\
 $\rightarrow$ 150+4 mod 35\\
 $\rightarrow$ 154 mod 35\\
$\therefore$ $y_3$ = 14  

Entonces ya sabemos el valor del punto 2Q, ahora debemos sumar Q+2q para tener 3Q, para eso debemos calcular de nuevo $\lambda$ por lo que haremos\vspace{.3cm}
$\lambda$= $\frac{-4-14} {15-0}$\vspace{.3cm}

Aquí encontramos un error debido a que 15 no tiene inverso multiplicativo en el grupo 35 así que eso implica que tenemos que sacar el MCD(35,15) = 5 $\therefore$ 5 es factor de 35.

\item[2] Factoriza 35 tratando de calcular 4Q duplicándolo.

Ahora no calcularemos 2Q debido a que ya lo calculamos anteriormente en el ejercicio 1 por lo cual pasaremos a calcular $\lambda$ bajo la definición del ejercicio 1\vspace{.3cm} 

$\lambda$= $\frac{3(0)^2}{2(14)}$mod 35 \vspace{.3cm}

Encontramos de nuevo el mismo problema que en el ejercicio anterior debido a que 28 mod 35  no tiene inverso, por lo cual debemos sacar su MCD(28,35) = 7 $\therefore$ 7 es un factor de 35 

\item[3] Calcula 3Q y 4Q sobre E (mod 5) y sobre E (mod 7) explica por que el factor 5
se obtiene calculando 3Q y por que el factor 7 se obtiene calculando 4Q.

Debido a que cuando calculas 3Q , intentamos sacar el inverso de 15 en el grupo 35, esto conflictuá ya que como 15 y 35 no son primos $\rightarrow$ que son números compuesto por primos esto nos lo sabemos por el teorema fundamental de la aritmética , ahora al sacar su MCD descubrimos que 5 es ese número $\therefore$ por esa razón 3 q , nos dio el valor 5 por compartir ese primo con 15 y análogamente pasa lo mismo con 28 y 35 

Ahora el valor de 3 Q con 5 =  No se puede calcular debido a que tenemos que cuando intentamos sumar Q= (15,-4) con 2Q=(0,4)(Los calculos de como se llego a 2q  se dejan como ejercicio para el lector ) e intentamos sacar $\alpha$ = (-4-4).(0-15)$^{-1}$\\
$\rightarrow$ 8 . (-15)$^{-1}$ y como -15 no esta en el campo de 5 entonces lo que hacemos es devolverlo con la operación modulo $\rightarrow$ -15 mod 5 = 0 y 0 no tiene inverso multiplicativo en 5 y no existe MCD(0,5) por lo que nuestro proceso termina aquí
  




\end{itemize}
\end{document}