\documentclass[10pt,a4paper]{article}
\usepackage[utf8]{inputenc}
\usepackage{amsmath}
\usepackage{amsfonts}
\usepackage{amssymb}
\usepackage{graphicx}
\usepackage[left=2cm,right=2cm,top=2cm,bottom=2cm]{geometry}
\author{Castro Mejia Jonatan Alejandro}
\title{Tarea 3 \\Criptografía y Seguridad \\ Curvas elípticas}
\begin{document}
\maketitle
\begin{enumerate}
\item Sea E: $y^{2}+20x= x^{3}+21(mod 35)$ y sea Q =(15,-4) $\in$ E.
\begin{itemize}
\item[a)] Factoriza 35 tratando de calcular 3Q.
\\
\item[b)] Factoriza 35 tratando de calcular 4Q duplicándolo.
\item[c)] Calcula 3Q y 4Q sobre E (mod 5) y sobre E (mod 7) explica por que el factor 5 se obtiene calculando 3Q y por que el factor 7 se obtiene calculando 4Q.

\end{itemize}
\item Sea E la curva elíptica $y^{2} =x^{3}+x+28 $ definida sobre $\mathbb{Z}_{71}$
\begin{itemize}
\item[a)] Calcula y muestra el número de puntos de E.\\
Puntos:\\ $O$, (1,32),
(1,39),
(2,31),
(2,40),
(3,22),
(3,49),
(4,5),
(4,66),
(5,4),
(5,67),
(6,26),
(6,45),
(12,8),
(12,63),
(13,26),
(13,45),
(15,9),
(15,62),
(19,27),
(19,44),
(20,5),
(20,66),
(21,3),
(21,68),
(22,30),
(22,41),
(23,19),
(23,52),
(25,22),
(25,49),
(27,0),
(31,32),
(31,39),
(33,1),
(33,70),
(34,23),
(34,48),
(35,14),
(35,57),
(36,12),
(36,59),
(37,33),
(37,38),
(39,32),
(39,39),
(41,7),
(41,64),
(43,22),
(43,49),
(47,5),
(47,66),
(48,11),
(48,60),
(49,24),
(49,47),
(52,26),
(52,45),
(53,0),
(58,27),
(58,44),
(61,15),
(61,56),
(62,0),
(63,17),
(63,54),
(65,27),
(65,44),
(66,18),
(66,53),
(69,35),
(69,36).


\item[b)] Muestra que E no es un grupo cíclico.
\item[c)] ¿Cuál es el máximo orden de un elemento en E? Encuentra un elemnto que tenga ese orden.
\end{itemize}
\item Sea E :$y^{2}-2=x^{3}+333x$ sobre $\mathbb{F}_{347}$ y sea P = (110,136).
\begin{itemize}
\item[a)] ¿Es Q=(81,-176) un punto de E?\\
Para verificar esto hay que sustituir en E: x = 81, y = -176.\\
(-176$)^{2} = (81)^{2} + 333(81) + 2 $\\
30976 = 6561 + 26973 + 2. \\
30976 = 33536 mod 347 \\
30976 $\neq$ 224 mod 347.\\
Entonces Q no es un punto de E.
\item[b)] si sabemos que $|E| = 358$ ¿Podemos decir E es criptográficamente útil?
¿Cuál es el orden de P? ¿Entre que valores se puede escojer la clave privada?
\item[c)] si tu clave privada es k=101 y algún conocido te ha enviado el mensaje cifrado (M$_1$=(232,278) y M$_2$=(135,214)) ¿Cuál era el mensaje original?
\end{itemize}
\item Sea $\mathbb{E}$ : F(x,y)=$y^{2}-x^{3}-2x-7$ sobre $\mathbb{Z}_{31}$ con $\neq \mathbb{E}$ = 39 y P = (2,9) es un punto de orden 39 sobre $\mathbb{E}$, el ECIES simplifado definido sobre $\mathbb{E}$ tiene $\mathbb{Z}^{*}_{31}$ como espacio de texto plano, supongamos que la clave privada es m = 8.
\begin{itemize}
\item[a)] Calcula Q=mP\\
Hay que calcular Q = 8P\\$ ~~~~~~~~~~~~~~~~~~~~~~~~~~~$= 4P + 4P = (2P+2P) + (2P+2P)\\\\
Como son los mismos puntos tenemos $\lambda$ = (3$x_{1}^{2}$ + A) $(2y_{1})^{-1}$ = $(3(2)^{2} + 2)(2(9))^{-1}$ hay que encontrar el inverso de 9 mod 31 usando el algoritmo extendido de euclides, 2(9)$^{-1}$ $\equiv$ 18$^{-1} \equiv $ 19 mod 31.\\
Entonces $\lambda$ = (12+2) x 19 = 266.\\
Queda calcular $x_{3}$ = $\lambda^{2} - x_{1} - x_{2}$, y$_3$ = $\lambda(x_{1} - x_{3}) - y_{1}$.
\\
\\x$_{3}$ = $(266)^{2}$ - 2 - 2 = 70,756-4 = 70,752 $\equiv$ 10 mod 31.
\\y$_{3}$ = 266(2-70752) - 9 = -18,819,509 $\equiv$  2 mod 31.\\
Entonces 2P = (10,2).\\
4P = 2P + 2P =  (10,2)+(10,2).\\
Entonces $\lambda$ = (3(10$^{2}$) + 2)(2(2)$)^{-1}$ = 2,416.\\
x$_{3}$ = (2,416)$^{2}$ -10 -10 = 5,837,036 $\equiv$ 15 mod 31.\\
y$_{3}$ = 2,416(10 - 5,837,036) - 2 = -14102254818 $\equiv$ 8 mod 31.\\
Por lo que 4P = (15,8), solo falta calcular 8P = 4P + 4P = (15,8) + (15,8).\\
Ahora $\lambda$ = (3(15)$^{2}$ + 2)(2(8)$^{-1}$); 2(8)$^{-1} \equiv $ 2 mod 31. \\
entonces $\lambda$ = 677 x 2 = 1354.\\
x$_{3}$ = 1,354$^{2}$ - 15 -15 = 1.833,286 $\equiv$ 8 mod 31.\\
y$_{3}$ = 1354(15 - 1,833,286 )-8 = -24,82,248,942 $\equiv$ 15 mod 31.\\
Entonces 8P = (8,15).
\item[b)] Descifra la siguiente cadena de texto cifrado: \\
((18,1),21),((3,1),18),((17,0),19),((28,0),8)\\
E : y$^{2} = x^{3} + 2x + 7$ mod 31
\begin{itemize}
\item[1)] ((18,1),21)\\
Evaluamos 18 en E:\\
Entonces 18$^{3} +2(18) + 7 = 5875 \equiv $16 mod 31.\\
y = $\pm$ 4, ahora hay que fijarnos en la segunda entrada la cual nos dice que y $\equiv$ 1 mod 2, entonces y = 27.\\
El punto de descompresión es (18,27), entonces 8(18,27) = (15,8)\\
Ahora hay que encontrar 15$^{-1} \equiv $ 29 mod 31,y con esto hay que calcular 29(21) mod 31 que nos da: 20.
\item[2)] ((3,1),18)\\
Evaluamos 3 en E :\\
Entonces 3$^{3} + 2(3) + 7 = 40 \equiv $ 9 mod 31\\
y = $\pm$ 3, ahora hay que fijarnos en la segunda entrada la cual nos dice que y $\equiv$ 1 mod 2, entonces y = 28 \\ 
El punto de descompresión es (3,28), entonces 8(3,28) = (2,22)\\
Ahora hay que encontrar 2$^{-1} \equiv$ 16 mod 31, y con esto hay que calcular 16(18) mod 31 que nos da: 9.
\item[3)] ((17,0),19)\\
Evaluamos 17 en E:\\
Entonces 17$^{3} + 2(17) + 7 = 4954 \equiv$ 25 mod 31\\
y = $\pm$ 5, ahora hay que fijarnos en la segunda entrada la cual nos dice que y $\equiv$ 0 mod 2, entonces y = 26\\
El Punto de descompresión es (17,26) ,entonces 8(17,26) = (29,29)\\
Ahora hay que encontrar 29$^{-1} \equiv$ 15 mod 31, y con esto hay que calcuar 15(19) mod 31 que nos da: 6
\item[4)] ((28,0),8) \\
Evaluamos 28 en E:\\
Entonces 28$^{3} + 2(28) + 7 = 22015 \equiv $5 mod 31
Hay que sumarle a 5, 31 tantas veces como sea necesario para que nos genere un cuadrado perfecto. En este caso 5 + 31 = 36.\\
Entonces y = $\pm$ 6, ahora nos fijamos en la segunda entrada la cual nos dice que y $\equiv$ 0 mod 2, entonces  y = 26\\
El punto de descompresión es (28,26), entonces 8(28,26) = (5,10)\\
Ahora hay que encontrar 5$^{-1} \equiv$ 25 mod 31, y con esto hay que calcular 25(8) mod 31 que nos da: 14.


\end{itemize}
\item[c)] Supongamos que cada texto plano representa un caracter alfabético, convierte el texto plano en una palabra en ingles, usa la asociación (A $\rightarrow$ 1, ... , Z $\rightarrow$ 26) en este caso 0 no es considerado como un texto plano o par ordenado.\\
Del ejercicio anterior obtuvimos los valores $\lbrace 20, 9,6,14 \rbrace$
\begin{center}
\begin{tabular}{|c|c|c|c|c|c|c|c|c|c|c|c|c|c|}
\hline
A & B & C & D & E & F & G & H & I & J & K & L & M \\
\hline
1 & 2 & 3 & 4 & 5 & 6 & 7 & 8 & 9 & 10 & 11 & 12 & 13\\
\hline 
N & O & P & Q & R & S & T & U & V & W & X & Y & Z \\
\hline
14 & 15 & 16 & 17 & 18 & 19 & 20 & 21 & 22 & 23 & 24 & 25 & 26\\
\hline 
\end{tabular}
\\Y con los valores obtenemos \textbf{TIFN} y si buscamos por acronimo, \\obtenemos \textbf{That's it for now}
\end{center}
\end{itemize}
\end{enumerate}
\end{document}